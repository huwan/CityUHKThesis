\begin{abstract}

Modern [research domain] systems face significant challenges in [problem area]. These systems have grown to unprecedented scale, with [key components] reaching [scale description]. However, the [limiting factor] often exceeds the capacity of typical [resource type]. Traditional approaches that rely on [traditional method] cannot support this scale. One potential solution is [proposed solution approach]. Nevertheless, naive implementation of this approach could significantly reduce system [performance metric] and introduce [performance issue].

This thesis aims to address the performance challenges of [system type] from a [perspective] perspective. The issue is mainly attributed to two factors: [factor 1] and [factor 2]. Additionally, introducing [component] makes it challenging for [system type] to maintain [performance requirement], such as [specific metric] because [technical explanation]. Even if only [specific scenario], it can negatively impact [system component]. The existing [current approach] is unaware of this problem and becomes suboptimal when handling [workload type].

Based on these observations, this thesis proposes [First System Name] as a novel [system type] architecture. It leverages [technology A] as the [component A] and [technology B] as the [component B] to improve overall performance. [First System Name] leverages [technique 1], [technique 2], and [technique 3] to improve the overall performance. The proposed [component description] supports both [functionality A] and [functionality B] by taking advantage of [technical approach]. The proposed [second component] further reduces the [resource usage] by applying [optimization technique].

This thesis also introduces [Second System Name], a [system type] that addresses the issue of [specific problem] in [application domain] using [technical approach]. The approach classifies [workload type] into two categories: [category A] and [category B], which are processed differently based on their [classification criteria]. For [category A], [Second System Name] uses [technique A] and [technique B] to optimize [process description]. For [category B], [Second System Name] utilizes [technique C] at multiple levels to efficiently [process description] while controlling [performance metric] by [optimization method].

We evaluate the performance of [First System Name] with our [implementation type] on [evaluation setup]. [First System Name] delivers up to X$\times$ [improvement type] than baseline solution in a [constraint] system. Our evaluation of [Second System Name] with [evaluation setup] shows that [Second System Name] reduces the [performance metric] by up to Y\% while maintaining the same [other metric] compared to the baseline.

In conclusion, this thesis aims to explore the potential of constructing [system type] using [technology] in scenarios where [constraint] is restricted. Our experimental results indicate that the methods proposed in this thesis can effectively tackle the issue of [problem description] in [application domain].

\end{abstract}
